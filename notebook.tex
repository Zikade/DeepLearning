
% Default to the notebook output style

    


% Inherit from the specified cell style.




    
\documentclass[11pt]{article}

    
    
    \usepackage[T1]{fontenc}
    % Nicer default font (+ math font) than Computer Modern for most use cases
    \usepackage{mathpazo}

    % Basic figure setup, for now with no caption control since it's done
    % automatically by Pandoc (which extracts ![](path) syntax from Markdown).
    \usepackage{graphicx}
    % We will generate all images so they have a width \maxwidth. This means
    % that they will get their normal width if they fit onto the page, but
    % are scaled down if they would overflow the margins.
    \makeatletter
    \def\maxwidth{\ifdim\Gin@nat@width>\linewidth\linewidth
    \else\Gin@nat@width\fi}
    \makeatother
    \let\Oldincludegraphics\includegraphics
    % Set max figure width to be 80% of text width, for now hardcoded.
    \renewcommand{\includegraphics}[1]{\Oldincludegraphics[width=.8\maxwidth]{#1}}
    % Ensure that by default, figures have no caption (until we provide a
    % proper Figure object with a Caption API and a way to capture that
    % in the conversion process - todo).
    \usepackage{caption}
    \DeclareCaptionLabelFormat{nolabel}{}
    \captionsetup{labelformat=nolabel}

    \usepackage{adjustbox} % Used to constrain images to a maximum size 
    \usepackage{xcolor} % Allow colors to be defined
    \usepackage{enumerate} % Needed for markdown enumerations to work
    \usepackage{geometry} % Used to adjust the document margins
    \usepackage{amsmath} % Equations
    \usepackage{amssymb} % Equations
    \usepackage{textcomp} % defines textquotesingle
    % Hack from http://tex.stackexchange.com/a/47451/13684:
    \AtBeginDocument{%
        \def\PYZsq{\textquotesingle}% Upright quotes in Pygmentized code
    }
    \usepackage{upquote} % Upright quotes for verbatim code
    \usepackage{eurosym} % defines \euro
    \usepackage[mathletters]{ucs} % Extended unicode (utf-8) support
    \usepackage[utf8x]{inputenc} % Allow utf-8 characters in the tex document
    \usepackage{fancyvrb} % verbatim replacement that allows latex
    \usepackage{grffile} % extends the file name processing of package graphics 
                         % to support a larger range 
    % The hyperref package gives us a pdf with properly built
    % internal navigation ('pdf bookmarks' for the table of contents,
    % internal cross-reference links, web links for URLs, etc.)
    \usepackage{hyperref}
    \usepackage{longtable} % longtable support required by pandoc >1.10
    \usepackage{booktabs}  % table support for pandoc > 1.12.2
    \usepackage[inline]{enumitem} % IRkernel/repr support (it uses the enumerate* environment)
    \usepackage[normalem]{ulem} % ulem is needed to support strikethroughs (\sout)
                                % normalem makes italics be italics, not underlines
    

    
    
    % Colors for the hyperref package
    \definecolor{urlcolor}{rgb}{0,.145,.698}
    \definecolor{linkcolor}{rgb}{.71,0.21,0.01}
    \definecolor{citecolor}{rgb}{.12,.54,.11}

    % ANSI colors
    \definecolor{ansi-black}{HTML}{3E424D}
    \definecolor{ansi-black-intense}{HTML}{282C36}
    \definecolor{ansi-red}{HTML}{E75C58}
    \definecolor{ansi-red-intense}{HTML}{B22B31}
    \definecolor{ansi-green}{HTML}{00A250}
    \definecolor{ansi-green-intense}{HTML}{007427}
    \definecolor{ansi-yellow}{HTML}{DDB62B}
    \definecolor{ansi-yellow-intense}{HTML}{B27D12}
    \definecolor{ansi-blue}{HTML}{208FFB}
    \definecolor{ansi-blue-intense}{HTML}{0065CA}
    \definecolor{ansi-magenta}{HTML}{D160C4}
    \definecolor{ansi-magenta-intense}{HTML}{A03196}
    \definecolor{ansi-cyan}{HTML}{60C6C8}
    \definecolor{ansi-cyan-intense}{HTML}{258F8F}
    \definecolor{ansi-white}{HTML}{C5C1B4}
    \definecolor{ansi-white-intense}{HTML}{A1A6B2}

    % commands and environments needed by pandoc snippets
    % extracted from the output of `pandoc -s`
    \providecommand{\tightlist}{%
      \setlength{\itemsep}{0pt}\setlength{\parskip}{0pt}}
    \DefineVerbatimEnvironment{Highlighting}{Verbatim}{commandchars=\\\{\}}
    % Add ',fontsize=\small' for more characters per line
    \newenvironment{Shaded}{}{}
    \newcommand{\KeywordTok}[1]{\textcolor[rgb]{0.00,0.44,0.13}{\textbf{{#1}}}}
    \newcommand{\DataTypeTok}[1]{\textcolor[rgb]{0.56,0.13,0.00}{{#1}}}
    \newcommand{\DecValTok}[1]{\textcolor[rgb]{0.25,0.63,0.44}{{#1}}}
    \newcommand{\BaseNTok}[1]{\textcolor[rgb]{0.25,0.63,0.44}{{#1}}}
    \newcommand{\FloatTok}[1]{\textcolor[rgb]{0.25,0.63,0.44}{{#1}}}
    \newcommand{\CharTok}[1]{\textcolor[rgb]{0.25,0.44,0.63}{{#1}}}
    \newcommand{\StringTok}[1]{\textcolor[rgb]{0.25,0.44,0.63}{{#1}}}
    \newcommand{\CommentTok}[1]{\textcolor[rgb]{0.38,0.63,0.69}{\textit{{#1}}}}
    \newcommand{\OtherTok}[1]{\textcolor[rgb]{0.00,0.44,0.13}{{#1}}}
    \newcommand{\AlertTok}[1]{\textcolor[rgb]{1.00,0.00,0.00}{\textbf{{#1}}}}
    \newcommand{\FunctionTok}[1]{\textcolor[rgb]{0.02,0.16,0.49}{{#1}}}
    \newcommand{\RegionMarkerTok}[1]{{#1}}
    \newcommand{\ErrorTok}[1]{\textcolor[rgb]{1.00,0.00,0.00}{\textbf{{#1}}}}
    \newcommand{\NormalTok}[1]{{#1}}
    
    % Additional commands for more recent versions of Pandoc
    \newcommand{\ConstantTok}[1]{\textcolor[rgb]{0.53,0.00,0.00}{{#1}}}
    \newcommand{\SpecialCharTok}[1]{\textcolor[rgb]{0.25,0.44,0.63}{{#1}}}
    \newcommand{\VerbatimStringTok}[1]{\textcolor[rgb]{0.25,0.44,0.63}{{#1}}}
    \newcommand{\SpecialStringTok}[1]{\textcolor[rgb]{0.73,0.40,0.53}{{#1}}}
    \newcommand{\ImportTok}[1]{{#1}}
    \newcommand{\DocumentationTok}[1]{\textcolor[rgb]{0.73,0.13,0.13}{\textit{{#1}}}}
    \newcommand{\AnnotationTok}[1]{\textcolor[rgb]{0.38,0.63,0.69}{\textbf{\textit{{#1}}}}}
    \newcommand{\CommentVarTok}[1]{\textcolor[rgb]{0.38,0.63,0.69}{\textbf{\textit{{#1}}}}}
    \newcommand{\VariableTok}[1]{\textcolor[rgb]{0.10,0.09,0.49}{{#1}}}
    \newcommand{\ControlFlowTok}[1]{\textcolor[rgb]{0.00,0.44,0.13}{\textbf{{#1}}}}
    \newcommand{\OperatorTok}[1]{\textcolor[rgb]{0.40,0.40,0.40}{{#1}}}
    \newcommand{\BuiltInTok}[1]{{#1}}
    \newcommand{\ExtensionTok}[1]{{#1}}
    \newcommand{\PreprocessorTok}[1]{\textcolor[rgb]{0.74,0.48,0.00}{{#1}}}
    \newcommand{\AttributeTok}[1]{\textcolor[rgb]{0.49,0.56,0.16}{{#1}}}
    \newcommand{\InformationTok}[1]{\textcolor[rgb]{0.38,0.63,0.69}{\textbf{\textit{{#1}}}}}
    \newcommand{\WarningTok}[1]{\textcolor[rgb]{0.38,0.63,0.69}{\textbf{\textit{{#1}}}}}
    
    
    % Define a nice break command that doesn't care if a line doesn't already
    % exist.
    \def\br{\hspace*{\fill} \\* }
    % Math Jax compatability definitions
    \def\gt{>}
    \def\lt{<}
    % Document parameters
    \title{2. Rosenblatt???}
    
    
    

    % Pygments definitions
    
\makeatletter
\def\PY@reset{\let\PY@it=\relax \let\PY@bf=\relax%
    \let\PY@ul=\relax \let\PY@tc=\relax%
    \let\PY@bc=\relax \let\PY@ff=\relax}
\def\PY@tok#1{\csname PY@tok@#1\endcsname}
\def\PY@toks#1+{\ifx\relax#1\empty\else%
    \PY@tok{#1}\expandafter\PY@toks\fi}
\def\PY@do#1{\PY@bc{\PY@tc{\PY@ul{%
    \PY@it{\PY@bf{\PY@ff{#1}}}}}}}
\def\PY#1#2{\PY@reset\PY@toks#1+\relax+\PY@do{#2}}

\expandafter\def\csname PY@tok@w\endcsname{\def\PY@tc##1{\textcolor[rgb]{0.73,0.73,0.73}{##1}}}
\expandafter\def\csname PY@tok@c\endcsname{\let\PY@it=\textit\def\PY@tc##1{\textcolor[rgb]{0.25,0.50,0.50}{##1}}}
\expandafter\def\csname PY@tok@cp\endcsname{\def\PY@tc##1{\textcolor[rgb]{0.74,0.48,0.00}{##1}}}
\expandafter\def\csname PY@tok@k\endcsname{\let\PY@bf=\textbf\def\PY@tc##1{\textcolor[rgb]{0.00,0.50,0.00}{##1}}}
\expandafter\def\csname PY@tok@kp\endcsname{\def\PY@tc##1{\textcolor[rgb]{0.00,0.50,0.00}{##1}}}
\expandafter\def\csname PY@tok@kt\endcsname{\def\PY@tc##1{\textcolor[rgb]{0.69,0.00,0.25}{##1}}}
\expandafter\def\csname PY@tok@o\endcsname{\def\PY@tc##1{\textcolor[rgb]{0.40,0.40,0.40}{##1}}}
\expandafter\def\csname PY@tok@ow\endcsname{\let\PY@bf=\textbf\def\PY@tc##1{\textcolor[rgb]{0.67,0.13,1.00}{##1}}}
\expandafter\def\csname PY@tok@nb\endcsname{\def\PY@tc##1{\textcolor[rgb]{0.00,0.50,0.00}{##1}}}
\expandafter\def\csname PY@tok@nf\endcsname{\def\PY@tc##1{\textcolor[rgb]{0.00,0.00,1.00}{##1}}}
\expandafter\def\csname PY@tok@nc\endcsname{\let\PY@bf=\textbf\def\PY@tc##1{\textcolor[rgb]{0.00,0.00,1.00}{##1}}}
\expandafter\def\csname PY@tok@nn\endcsname{\let\PY@bf=\textbf\def\PY@tc##1{\textcolor[rgb]{0.00,0.00,1.00}{##1}}}
\expandafter\def\csname PY@tok@ne\endcsname{\let\PY@bf=\textbf\def\PY@tc##1{\textcolor[rgb]{0.82,0.25,0.23}{##1}}}
\expandafter\def\csname PY@tok@nv\endcsname{\def\PY@tc##1{\textcolor[rgb]{0.10,0.09,0.49}{##1}}}
\expandafter\def\csname PY@tok@no\endcsname{\def\PY@tc##1{\textcolor[rgb]{0.53,0.00,0.00}{##1}}}
\expandafter\def\csname PY@tok@nl\endcsname{\def\PY@tc##1{\textcolor[rgb]{0.63,0.63,0.00}{##1}}}
\expandafter\def\csname PY@tok@ni\endcsname{\let\PY@bf=\textbf\def\PY@tc##1{\textcolor[rgb]{0.60,0.60,0.60}{##1}}}
\expandafter\def\csname PY@tok@na\endcsname{\def\PY@tc##1{\textcolor[rgb]{0.49,0.56,0.16}{##1}}}
\expandafter\def\csname PY@tok@nt\endcsname{\let\PY@bf=\textbf\def\PY@tc##1{\textcolor[rgb]{0.00,0.50,0.00}{##1}}}
\expandafter\def\csname PY@tok@nd\endcsname{\def\PY@tc##1{\textcolor[rgb]{0.67,0.13,1.00}{##1}}}
\expandafter\def\csname PY@tok@s\endcsname{\def\PY@tc##1{\textcolor[rgb]{0.73,0.13,0.13}{##1}}}
\expandafter\def\csname PY@tok@sd\endcsname{\let\PY@it=\textit\def\PY@tc##1{\textcolor[rgb]{0.73,0.13,0.13}{##1}}}
\expandafter\def\csname PY@tok@si\endcsname{\let\PY@bf=\textbf\def\PY@tc##1{\textcolor[rgb]{0.73,0.40,0.53}{##1}}}
\expandafter\def\csname PY@tok@se\endcsname{\let\PY@bf=\textbf\def\PY@tc##1{\textcolor[rgb]{0.73,0.40,0.13}{##1}}}
\expandafter\def\csname PY@tok@sr\endcsname{\def\PY@tc##1{\textcolor[rgb]{0.73,0.40,0.53}{##1}}}
\expandafter\def\csname PY@tok@ss\endcsname{\def\PY@tc##1{\textcolor[rgb]{0.10,0.09,0.49}{##1}}}
\expandafter\def\csname PY@tok@sx\endcsname{\def\PY@tc##1{\textcolor[rgb]{0.00,0.50,0.00}{##1}}}
\expandafter\def\csname PY@tok@m\endcsname{\def\PY@tc##1{\textcolor[rgb]{0.40,0.40,0.40}{##1}}}
\expandafter\def\csname PY@tok@gh\endcsname{\let\PY@bf=\textbf\def\PY@tc##1{\textcolor[rgb]{0.00,0.00,0.50}{##1}}}
\expandafter\def\csname PY@tok@gu\endcsname{\let\PY@bf=\textbf\def\PY@tc##1{\textcolor[rgb]{0.50,0.00,0.50}{##1}}}
\expandafter\def\csname PY@tok@gd\endcsname{\def\PY@tc##1{\textcolor[rgb]{0.63,0.00,0.00}{##1}}}
\expandafter\def\csname PY@tok@gi\endcsname{\def\PY@tc##1{\textcolor[rgb]{0.00,0.63,0.00}{##1}}}
\expandafter\def\csname PY@tok@gr\endcsname{\def\PY@tc##1{\textcolor[rgb]{1.00,0.00,0.00}{##1}}}
\expandafter\def\csname PY@tok@ge\endcsname{\let\PY@it=\textit}
\expandafter\def\csname PY@tok@gs\endcsname{\let\PY@bf=\textbf}
\expandafter\def\csname PY@tok@gp\endcsname{\let\PY@bf=\textbf\def\PY@tc##1{\textcolor[rgb]{0.00,0.00,0.50}{##1}}}
\expandafter\def\csname PY@tok@go\endcsname{\def\PY@tc##1{\textcolor[rgb]{0.53,0.53,0.53}{##1}}}
\expandafter\def\csname PY@tok@gt\endcsname{\def\PY@tc##1{\textcolor[rgb]{0.00,0.27,0.87}{##1}}}
\expandafter\def\csname PY@tok@err\endcsname{\def\PY@bc##1{\setlength{\fboxsep}{0pt}\fcolorbox[rgb]{1.00,0.00,0.00}{1,1,1}{\strut ##1}}}
\expandafter\def\csname PY@tok@kc\endcsname{\let\PY@bf=\textbf\def\PY@tc##1{\textcolor[rgb]{0.00,0.50,0.00}{##1}}}
\expandafter\def\csname PY@tok@kd\endcsname{\let\PY@bf=\textbf\def\PY@tc##1{\textcolor[rgb]{0.00,0.50,0.00}{##1}}}
\expandafter\def\csname PY@tok@kn\endcsname{\let\PY@bf=\textbf\def\PY@tc##1{\textcolor[rgb]{0.00,0.50,0.00}{##1}}}
\expandafter\def\csname PY@tok@kr\endcsname{\let\PY@bf=\textbf\def\PY@tc##1{\textcolor[rgb]{0.00,0.50,0.00}{##1}}}
\expandafter\def\csname PY@tok@bp\endcsname{\def\PY@tc##1{\textcolor[rgb]{0.00,0.50,0.00}{##1}}}
\expandafter\def\csname PY@tok@fm\endcsname{\def\PY@tc##1{\textcolor[rgb]{0.00,0.00,1.00}{##1}}}
\expandafter\def\csname PY@tok@vc\endcsname{\def\PY@tc##1{\textcolor[rgb]{0.10,0.09,0.49}{##1}}}
\expandafter\def\csname PY@tok@vg\endcsname{\def\PY@tc##1{\textcolor[rgb]{0.10,0.09,0.49}{##1}}}
\expandafter\def\csname PY@tok@vi\endcsname{\def\PY@tc##1{\textcolor[rgb]{0.10,0.09,0.49}{##1}}}
\expandafter\def\csname PY@tok@vm\endcsname{\def\PY@tc##1{\textcolor[rgb]{0.10,0.09,0.49}{##1}}}
\expandafter\def\csname PY@tok@sa\endcsname{\def\PY@tc##1{\textcolor[rgb]{0.73,0.13,0.13}{##1}}}
\expandafter\def\csname PY@tok@sb\endcsname{\def\PY@tc##1{\textcolor[rgb]{0.73,0.13,0.13}{##1}}}
\expandafter\def\csname PY@tok@sc\endcsname{\def\PY@tc##1{\textcolor[rgb]{0.73,0.13,0.13}{##1}}}
\expandafter\def\csname PY@tok@dl\endcsname{\def\PY@tc##1{\textcolor[rgb]{0.73,0.13,0.13}{##1}}}
\expandafter\def\csname PY@tok@s2\endcsname{\def\PY@tc##1{\textcolor[rgb]{0.73,0.13,0.13}{##1}}}
\expandafter\def\csname PY@tok@sh\endcsname{\def\PY@tc##1{\textcolor[rgb]{0.73,0.13,0.13}{##1}}}
\expandafter\def\csname PY@tok@s1\endcsname{\def\PY@tc##1{\textcolor[rgb]{0.73,0.13,0.13}{##1}}}
\expandafter\def\csname PY@tok@mb\endcsname{\def\PY@tc##1{\textcolor[rgb]{0.40,0.40,0.40}{##1}}}
\expandafter\def\csname PY@tok@mf\endcsname{\def\PY@tc##1{\textcolor[rgb]{0.40,0.40,0.40}{##1}}}
\expandafter\def\csname PY@tok@mh\endcsname{\def\PY@tc##1{\textcolor[rgb]{0.40,0.40,0.40}{##1}}}
\expandafter\def\csname PY@tok@mi\endcsname{\def\PY@tc##1{\textcolor[rgb]{0.40,0.40,0.40}{##1}}}
\expandafter\def\csname PY@tok@il\endcsname{\def\PY@tc##1{\textcolor[rgb]{0.40,0.40,0.40}{##1}}}
\expandafter\def\csname PY@tok@mo\endcsname{\def\PY@tc##1{\textcolor[rgb]{0.40,0.40,0.40}{##1}}}
\expandafter\def\csname PY@tok@ch\endcsname{\let\PY@it=\textit\def\PY@tc##1{\textcolor[rgb]{0.25,0.50,0.50}{##1}}}
\expandafter\def\csname PY@tok@cm\endcsname{\let\PY@it=\textit\def\PY@tc##1{\textcolor[rgb]{0.25,0.50,0.50}{##1}}}
\expandafter\def\csname PY@tok@cpf\endcsname{\let\PY@it=\textit\def\PY@tc##1{\textcolor[rgb]{0.25,0.50,0.50}{##1}}}
\expandafter\def\csname PY@tok@c1\endcsname{\let\PY@it=\textit\def\PY@tc##1{\textcolor[rgb]{0.25,0.50,0.50}{##1}}}
\expandafter\def\csname PY@tok@cs\endcsname{\let\PY@it=\textit\def\PY@tc##1{\textcolor[rgb]{0.25,0.50,0.50}{##1}}}

\def\PYZbs{\char`\\}
\def\PYZus{\char`\_}
\def\PYZob{\char`\{}
\def\PYZcb{\char`\}}
\def\PYZca{\char`\^}
\def\PYZam{\char`\&}
\def\PYZlt{\char`\<}
\def\PYZgt{\char`\>}
\def\PYZsh{\char`\#}
\def\PYZpc{\char`\%}
\def\PYZdl{\char`\$}
\def\PYZhy{\char`\-}
\def\PYZsq{\char`\'}
\def\PYZdq{\char`\"}
\def\PYZti{\char`\~}
% for compatibility with earlier versions
\def\PYZat{@}
\def\PYZlb{[}
\def\PYZrb{]}
\makeatother


    % Exact colors from NB
    \definecolor{incolor}{rgb}{0.0, 0.0, 0.5}
    \definecolor{outcolor}{rgb}{0.545, 0.0, 0.0}



    
    % Prevent overflowing lines due to hard-to-break entities
    \sloppy 
    % Setup hyperref package
    \hypersetup{
      breaklinks=true,  % so long urls are correctly broken across lines
      colorlinks=true,
      urlcolor=urlcolor,
      linkcolor=linkcolor,
      citecolor=citecolor,
      }
    % Slightly bigger margins than the latex defaults
    
    \geometry{verbose,tmargin=1in,bmargin=1in,lmargin=1in,rmargin=1in}
    
    

    \begin{document}
    
    
    \maketitle
    
    

    
    本章分为如下内容:

\begin{itemize}
\tightlist
\item
  Rosenblatt感知器简介
\item
  感知器收敛算法
\item
  感知器实现
\item
  Rosenblatt感知器尝试解决线性不可分问题
\end{itemize}

    \section{2.1
Rosenblatt感知器简介}\label{rosenblattux611fux77e5ux5668ux7b80ux4ecb}

在神经网络的形成阶段(1943-1958),一些研究者做出了开拓性的贡献:

\begin{itemize}
\tightlist
\item
  McCulloch and Pitts(1943)引入神经网络的概念,并设计了M-P模型。
\item
  Hebb(1949)提出自组织学习的第一个规则。
\item
  Rosenblatt(1958)提出感知器作为神经网络中监督学习的第一个模型。
\end{itemize}

感知器是用于线性可分模式(即类别分别位于超平面分割的两边)分类的最简单的神经网络模型。它在M-P模型的基础上添加了训练方法,并且Rosenblatt证明了当用来训练感知器的模式(向量)取自两个线性可分的类时,感知器算法是收敛的。算法的收敛性证明被称为感知器收敛定理。

最简单的感知器由一个M-P神经元构成,只能完成两类的模式分类,通过扩展感知器的输出层,相应的可以完成多于两类的分类。但是,只有这些类是线性可分时感知器才能正常工作。Rosenblatt感知器是建立在一个非线性的M-P神经元上。

但需要注意的是,Rosenblatt感知器的激活函数形式与M-P模型有所不同。在M-P模型中激活函数输出
``0'',表示抑制,在Rosenblatt感知器中激活函数输出
``-1'',表示神经元抑制。即激活函数形式为:

\[\begin{equation}\mathrm{sgn}(x)=\begin{cases}1, & x \geq 0\\ -1, & x < 0 \end{cases}\end{equation}\]

    上述模型中 \(x_1, x_2,...,x_n\)
表示外部刺激,也可记为\(\mathbf x\),也就是模型输入,\(w_1, w_2,..., w_n\)
为突触权值,也就是模型自由参数,也可记为\(\mathbf w\),\(b\)
表示阈值。为了简化表示,令 \(x_0 = 1, w_0=-b\),则模型输入可表示为
\(\mathbf{w}^\mathrm{T}\mathbf{x}\)。即Rosenblatt感知器表示为\(\mathrm{sgn}(\mathbf{w}^\mathrm{T}\mathbf{x})\)。

注意:当我们把感知器中的参数\(b\)看做是阈值时,往往使用\(y=f\left(\sum_{i=1}^nw_ix_i-b\right)\)表示模型,注意这时候\(b\)前面使用的是负号;当我们把\(b\)看作是偏置值时,往往使用\(y=f\left(\sum_{i=0}^nw_ix_i\right)\)表示模型,其中\(x_0=1\),二者是等价的。

    当\(\mathbf{w}^\mathrm{T}\mathbf{x}\geq 0\),则模型输出
``1'',表示输出某一类,此处表示正类,否则输出
``-1'',表示负类。即\(\mathbf{w}^\mathrm{T}\mathbf{x}= 0\)表示决策边界。例如在二维两类模式分类问题中,决策边界为\(w_0+w_1x_1+w_2x_2=0\),即为二维平面坐标系中的一条直线,如下图所示。

    \begin{Verbatim}[commandchars=\\\{\}]
{\color{incolor}In [{\color{incolor}2}]:} \PY{o}{\PYZpc{}}\PY{k}{matplotlib} inline
        \PY{k+kn}{import} \PY{n+nn}{numpy} \PY{k}{as} \PY{n+nn}{np}
        \PY{k+kn}{from} \PY{n+nn}{matplotlib} \PY{k}{import} \PY{n}{pyplot} \PY{k}{as} \PY{n}{plt}
        
        \PY{n}{plt}\PY{o}{.}\PY{n}{figure}\PY{p}{(}\PY{n}{figsize}\PY{o}{=}\PY{p}{(}\PY{l+m+mi}{8}\PY{p}{,} \PY{l+m+mi}{5}\PY{p}{)}\PY{p}{,} \PY{n}{dpi}\PY{o}{=}\PY{l+m+mi}{100}\PY{p}{)}
        
        \PY{n}{X} \PY{o}{=} \PY{n}{np}\PY{o}{.}\PY{n}{linspace}\PY{p}{(}\PY{o}{\PYZhy{}}\PY{l+m+mi}{5}\PY{p}{,} \PY{l+m+mi}{5}\PY{p}{,} \PY{l+m+mi}{256}\PY{p}{)}
        \PY{n}{Y} \PY{o}{=} \PY{n}{X} \PY{o}{*} \PY{p}{(}\PY{o}{\PYZhy{}}\PY{l+m+mf}{1.2}\PY{p}{)} \PY{o}{+} \PY{l+m+mf}{2.1}
        \PY{n}{ax} \PY{o}{=} \PY{n}{plt}\PY{o}{.}\PY{n}{gca}\PY{p}{(}\PY{p}{)}
        \PY{n}{ax}\PY{o}{.}\PY{n}{spines}\PY{p}{[}\PY{l+s+s1}{\PYZsq{}}\PY{l+s+s1}{right}\PY{l+s+s1}{\PYZsq{}}\PY{p}{]}\PY{o}{.}\PY{n}{set\PYZus{}color}\PY{p}{(}\PY{l+s+s1}{\PYZsq{}}\PY{l+s+s1}{none}\PY{l+s+s1}{\PYZsq{}}\PY{p}{)}
        \PY{n}{ax}\PY{o}{.}\PY{n}{spines}\PY{p}{[}\PY{l+s+s1}{\PYZsq{}}\PY{l+s+s1}{top}\PY{l+s+s1}{\PYZsq{}}\PY{p}{]}\PY{o}{.}\PY{n}{set\PYZus{}color}\PY{p}{(}\PY{l+s+s1}{\PYZsq{}}\PY{l+s+s1}{none}\PY{l+s+s1}{\PYZsq{}}\PY{p}{)}
        \PY{n}{ax}\PY{o}{.}\PY{n}{spines}\PY{p}{[}\PY{l+s+s1}{\PYZsq{}}\PY{l+s+s1}{left}\PY{l+s+s1}{\PYZsq{}}\PY{p}{]}\PY{o}{.}\PY{n}{set\PYZus{}position}\PY{p}{(}\PY{p}{(}\PY{l+s+s1}{\PYZsq{}}\PY{l+s+s1}{data}\PY{l+s+s1}{\PYZsq{}}\PY{p}{,} \PY{l+m+mi}{0}\PY{p}{)}\PY{p}{)}
        \PY{n}{ax}\PY{o}{.}\PY{n}{spines}\PY{p}{[}\PY{l+s+s1}{\PYZsq{}}\PY{l+s+s1}{bottom}\PY{l+s+s1}{\PYZsq{}}\PY{p}{]}\PY{o}{.}\PY{n}{set\PYZus{}position}\PY{p}{(}\PY{p}{(}\PY{l+s+s1}{\PYZsq{}}\PY{l+s+s1}{data}\PY{l+s+s1}{\PYZsq{}}\PY{p}{,} \PY{l+m+mi}{0}\PY{p}{)}\PY{p}{)}
        
        \PY{n}{plt}\PY{o}{.}\PY{n}{plot}\PY{p}{(}\PY{n}{X}\PY{p}{,} \PY{n}{Y}\PY{p}{,} \PY{n}{color}\PY{o}{=}\PY{l+s+s1}{\PYZsq{}}\PY{l+s+s1}{black}\PY{l+s+s1}{\PYZsq{}}\PY{p}{,} \PY{n}{linewidth}\PY{o}{=}\PY{l+m+mi}{2}\PY{p}{)}
        \PY{n}{plt}\PY{o}{.}\PY{n}{fill\PYZus{}between}\PY{p}{(}\PY{n}{X}\PY{p}{,} \PY{n}{Y}\PY{o}{.}\PY{n}{min}\PY{p}{(}\PY{p}{)}\PY{p}{,} \PY{n}{Y}\PY{p}{,} \PY{n}{facecolor}\PY{o}{=}\PY{l+s+s1}{\PYZsq{}}\PY{l+s+s1}{red}\PY{l+s+s1}{\PYZsq{}}\PY{p}{,} \PY{n}{alpha}\PY{o}{=}\PY{l+s+s1}{\PYZsq{}}\PY{l+s+s1}{0.3}\PY{l+s+s1}{\PYZsq{}}\PY{p}{)}
        \PY{n}{plt}\PY{o}{.}\PY{n}{fill\PYZus{}between}\PY{p}{(}\PY{n}{X}\PY{p}{,} \PY{n}{Y}\PY{p}{,} \PY{n}{Y}\PY{o}{.}\PY{n}{max}\PY{p}{(}\PY{p}{)}\PY{p}{,} \PY{n}{facecolor}\PY{o}{=}\PY{l+s+s1}{\PYZsq{}}\PY{l+s+s1}{blue}\PY{l+s+s1}{\PYZsq{}}\PY{p}{,} \PY{n}{alpha}\PY{o}{=}\PY{l+s+s1}{\PYZsq{}}\PY{l+s+s1}{0.3}\PY{l+s+s1}{\PYZsq{}}\PY{p}{)}
        \PY{n}{plt}\PY{o}{.}\PY{n}{annotate}\PY{p}{(}
            \PY{l+s+s1}{\PYZsq{}}\PY{l+s+s1}{决策边界}\PY{l+s+s1}{\PYZsq{}}\PY{p}{,} \PY{n}{xy}\PY{o}{=}\PY{p}{(}\PY{o}{\PYZhy{}}\PY{l+m+mi}{3}\PY{p}{,} \PY{l+m+mi}{3} \PY{o}{*} \PY{l+m+mf}{1.2} \PY{o}{+} \PY{l+m+mf}{2.1}\PY{p}{)}\PY{p}{,} \PY{n}{xytext}\PY{o}{=}\PY{p}{(}\PY{o}{\PYZhy{}}\PY{l+m+mf}{2.}\PY{p}{,} \PY{o}{+}\PY{l+m+mf}{6.5}\PY{p}{)}\PY{p}{,} 
             \PY{n}{arrowprops}\PY{o}{=}\PY{n+nb}{dict}\PY{p}{(}\PY{n}{arrowstyle}\PY{o}{=}\PY{l+s+s2}{\PYZdq{}}\PY{l+s+s2}{\PYZhy{}\PYZgt{}}\PY{l+s+s2}{\PYZdq{}}\PY{p}{,} \PY{n}{connectionstyle}\PY{o}{=}\PY{l+s+s2}{\PYZdq{}}\PY{l+s+s2}{arc3,rad=.2}\PY{l+s+s2}{\PYZdq{}}\PY{p}{)}\PY{p}{,} 
             \PY{n}{fontproperties}\PY{o}{=}\PY{l+s+s1}{\PYZsq{}}\PY{l+s+s1}{SimHei}\PY{l+s+s1}{\PYZsq{}}\PY{p}{,} \PY{n}{fontsize}\PY{o}{=}\PY{l+m+mi}{14}\PY{p}{)}
        
        \PY{n}{plt}\PY{o}{.}\PY{n}{text}\PY{p}{(}\PY{l+m+mi}{2}\PY{p}{,} \PY{l+m+mf}{6.5}\PY{p}{,} \PY{l+s+s1}{\PYZsq{}}\PY{l+s+s1}{正类}\PY{l+s+s1}{\PYZsq{}}\PY{p}{,} \PY{n}{family} \PY{o}{=} \PY{l+s+s2}{\PYZdq{}}\PY{l+s+s2}{SimSun}\PY{l+s+s2}{\PYZdq{}}\PY{p}{,} \PY{n}{size}\PY{o}{=}\PY{l+m+mi}{14}\PY{p}{)}
        \PY{n}{plt}\PY{o}{.}\PY{n}{text}\PY{p}{(}\PY{o}{\PYZhy{}}\PY{l+m+mi}{3}\PY{p}{,} \PY{l+m+mi}{2}\PY{p}{,} \PY{l+s+s1}{\PYZsq{}}\PY{l+s+s1}{负类}\PY{l+s+s1}{\PYZsq{}}\PY{p}{,} \PY{n}{family} \PY{o}{=} \PY{l+s+s2}{\PYZdq{}}\PY{l+s+s2}{SimSun}\PY{l+s+s2}{\PYZdq{}}\PY{p}{,} \PY{n}{size}\PY{o}{=}\PY{l+m+mi}{14}\PY{p}{)}
        
        \PY{n}{plt}\PY{o}{.}\PY{n}{show}\PY{p}{(}\PY{p}{)}
\end{Verbatim}


    \begin{center}
    \adjustimage{max size={0.9\linewidth}{0.9\paperheight}}{output_4_0.png}
    \end{center}
    { \hspace*{\fill} \\}
    
    \section{2.2
感知器收敛算法}\label{ux611fux77e5ux5668ux6536ux655bux7b97ux6cd5}

对于线性可分模式,设置合理的感知器参数一定能使得感知器正确分类数据,感知器参数的设置被称为感知器收敛算法,算法流程如下:

    感知器收敛算法的更新权值规则中,\(d(n)-y(n)\)用于衡量误差,学习率\(\eta\)起到调节权值更新速率的作用,学习率较大,参数更新速度快,但可能影响收敛,学习率较小,收敛平稳,但速度较慢。

    \section{2.3
Rosenblatt感知器实现}\label{rosenblattux611fux77e5ux5668ux5b9eux73b0}

实现Rosenblatt感知器与收敛算法,并使用均方误差(MSE)衡量模型性能。这里使用线性可分的双月数据集进行训练与测试。

    双月数据集是一个非对称面对面的``月亮''区域组成的数据集。落在上半月的数据集为一个类别,落在下半月的数据集是另一个类别,每个半月拥有相同的半径\(r\)、厚度\(w\),两个月亮之间的垂直距离为\(d\),其中\(d>0\)表示两个月亮是分离的,即线性可分的,否则则为线性不可分的。

    这里我们使用如下算法生成双月数据集:

    \begin{Verbatim}[commandchars=\\\{\}]
{\color{incolor}In [{\color{incolor}3}]:} \PY{k}{def} \PY{n+nf}{dbmoon}\PY{p}{(}\PY{n}{N}\PY{o}{=}\PY{l+m+mi}{100}\PY{p}{,} \PY{n}{d}\PY{o}{=}\PY{l+m+mi}{2}\PY{p}{,} \PY{n}{r}\PY{o}{=}\PY{l+m+mi}{10}\PY{p}{,} \PY{n}{w}\PY{o}{=}\PY{l+m+mi}{2}\PY{p}{)}\PY{p}{:}
            \PY{l+s+sd}{\PYZsq{}\PYZsq{}\PYZsq{}生成双月数据集}
        \PY{l+s+sd}{    }
        \PY{l+s+sd}{    Args:}
        \PY{l+s+sd}{        N: 数据集数量}
        \PY{l+s+sd}{        d: 双月之间的距离}
        \PY{l+s+sd}{        r: 双月半径}
        \PY{l+s+sd}{        w: 每个月亮的厚度}
        \PY{l+s+sd}{        }
        \PY{l+s+sd}{    Returns:}
        \PY{l+s+sd}{        返回一个`shape=[N * 2, 2]`的数组,前100个元素表示}
        \PY{l+s+sd}{        上半月,后100个元素表示下半月。}
        \PY{l+s+sd}{    \PYZsq{}\PYZsq{}\PYZsq{}}
            \PY{n}{w2} \PY{o}{=} \PY{n}{w} \PY{o}{/} \PY{l+m+mi}{2}
            \PY{n}{data} \PY{o}{=} \PY{n}{np}\PY{o}{.}\PY{n}{empty}\PY{p}{(}\PY{l+m+mi}{0}\PY{p}{)}
            \PY{k}{while} \PY{n}{data}\PY{o}{.}\PY{n}{shape}\PY{p}{[}\PY{l+m+mi}{0}\PY{p}{]} \PY{o}{\PYZlt{}} \PY{n}{N}\PY{p}{:}
                \PY{c+c1}{\PYZsh{}generate Rectangular data}
                \PY{n}{tmp\PYZus{}x} \PY{o}{=} \PY{l+m+mi}{2} \PY{o}{*} \PY{p}{(}\PY{n}{r} \PY{o}{+} \PY{n}{w2}\PY{p}{)} \PY{o}{*} \PY{p}{(}\PY{n}{np}\PY{o}{.}\PY{n}{random}\PY{o}{.}\PY{n}{random}\PY{p}{(}\PY{p}{[}\PY{n}{N}\PY{p}{,} \PY{l+m+mi}{1}\PY{p}{]}\PY{p}{)} \PY{o}{\PYZhy{}} \PY{l+m+mf}{0.5}\PY{p}{)}
                \PY{n}{tmp\PYZus{}y} \PY{o}{=} \PY{p}{(}\PY{n}{r} \PY{o}{+} \PY{n}{w2}\PY{p}{)} \PY{o}{*} \PY{n}{np}\PY{o}{.}\PY{n}{random}\PY{o}{.}\PY{n}{random}\PY{p}{(}\PY{p}{[}\PY{n}{N}\PY{p}{,} \PY{l+m+mi}{1}\PY{p}{]}\PY{p}{)}
                \PY{n}{tmp} \PY{o}{=} \PY{n}{np}\PY{o}{.}\PY{n}{concatenate}\PY{p}{(}\PY{p}{(}\PY{n}{tmp\PYZus{}x}\PY{p}{,} \PY{n}{tmp\PYZus{}y}\PY{p}{)}\PY{p}{,} \PY{n}{axis}\PY{o}{=}\PY{l+m+mi}{1}\PY{p}{)}
                \PY{n}{tmp\PYZus{}ds} \PY{o}{=} \PY{n}{np}\PY{o}{.}\PY{n}{sqrt}\PY{p}{(}\PY{n}{tmp\PYZus{}x} \PY{o}{*} \PY{n}{tmp\PYZus{}x} \PY{o}{+} \PY{n}{tmp\PYZus{}y} \PY{o}{*} \PY{n}{tmp\PYZus{}y}\PY{p}{)}
                \PY{c+c1}{\PYZsh{}generate double moon data \PYZhy{}\PYZhy{}\PYZhy{}upper}
                \PY{n}{idx} \PY{o}{=} \PY{n}{np}\PY{o}{.}\PY{n}{logical\PYZus{}and}\PY{p}{(}\PY{n}{tmp\PYZus{}ds} \PY{o}{\PYZgt{}} \PY{p}{(}\PY{n}{r} \PY{o}{\PYZhy{}} \PY{n}{w2}\PY{p}{)}\PY{p}{,} \PY{n}{tmp\PYZus{}ds} \PY{o}{\PYZlt{}} \PY{p}{(}\PY{n}{r} \PY{o}{+} \PY{n}{w2}\PY{p}{)}\PY{p}{)}
                \PY{n}{idx} \PY{o}{=} \PY{p}{(}\PY{n}{idx}\PY{o}{.}\PY{n}{nonzero}\PY{p}{(}\PY{p}{)}\PY{p}{)}\PY{p}{[}\PY{l+m+mi}{0}\PY{p}{]}
        
                \PY{k}{if} \PY{n}{data}\PY{o}{.}\PY{n}{shape}\PY{p}{[}\PY{l+m+mi}{0}\PY{p}{]} \PY{o}{==} \PY{l+m+mi}{0}\PY{p}{:}
                    \PY{n}{data} \PY{o}{=} \PY{n}{tmp}\PY{o}{.}\PY{n}{take}\PY{p}{(}\PY{n}{idx}\PY{p}{,} \PY{n}{axis}\PY{o}{=}\PY{l+m+mi}{0}\PY{p}{)}
                \PY{k}{else}\PY{p}{:}
                    \PY{n}{data} \PY{o}{=} \PY{n}{np}\PY{o}{.}\PY{n}{concatenate}\PY{p}{(}\PY{p}{(}\PY{n}{data}\PY{p}{,} \PY{n}{tmp}\PY{o}{.}\PY{n}{take}\PY{p}{(}\PY{n}{idx}\PY{p}{,} \PY{n}{axis}\PY{o}{=}\PY{l+m+mi}{0}\PY{p}{)}\PY{p}{)}\PY{p}{,} \PY{n}{axis}\PY{o}{=}\PY{l+m+mi}{0}\PY{p}{)}
            
            \PY{n}{db\PYZus{}moon} \PY{o}{=} \PY{n}{data}\PY{p}{[}\PY{l+m+mi}{0}\PY{p}{:} \PY{n}{N}\PY{p}{,} \PY{p}{:}\PY{p}{]}
            \PY{c+c1}{\PYZsh{}generate double moon data \PYZhy{}\PYZhy{}\PYZhy{}\PYZhy{}down}
            \PY{n}{data\PYZus{}t} \PY{o}{=} \PY{n}{np}\PY{o}{.}\PY{n}{empty}\PY{p}{(}\PY{p}{[}\PY{n}{N}\PY{p}{,} \PY{l+m+mi}{2}\PY{p}{]}\PY{p}{)}
            \PY{n}{data\PYZus{}t}\PY{p}{[}\PY{p}{:}\PY{p}{,} \PY{l+m+mi}{0}\PY{p}{]} \PY{o}{=} \PY{n}{data}\PY{p}{[}\PY{l+m+mi}{0}\PY{p}{:} \PY{n}{N}\PY{p}{,} \PY{l+m+mi}{0}\PY{p}{]} \PY{o}{+} \PY{n}{r}
            \PY{n}{data\PYZus{}t}\PY{p}{[}\PY{p}{:}\PY{p}{,} \PY{l+m+mi}{1}\PY{p}{]} \PY{o}{=} \PY{o}{\PYZhy{}}\PY{n}{data}\PY{p}{[}\PY{l+m+mi}{0}\PY{p}{:} \PY{n}{N}\PY{p}{,} \PY{l+m+mi}{1}\PY{p}{]} \PY{o}{\PYZhy{}} \PY{n}{d}
            \PY{n}{db\PYZus{}moon} \PY{o}{=} \PY{n}{np}\PY{o}{.}\PY{n}{concatenate}\PY{p}{(}\PY{p}{(}\PY{n}{db\PYZus{}moon}\PY{p}{,} \PY{n}{data\PYZus{}t}\PY{p}{)}\PY{p}{,} \PY{n}{axis}\PY{o}{=}\PY{l+m+mi}{0}\PY{p}{)}
            \PY{k}{return} \PY{n}{db\PYZus{}moon}
\end{Verbatim}


    生成2500个数据样本,并对数据样本进行可视化。

    \begin{Verbatim}[commandchars=\\\{\}]
{\color{incolor}In [{\color{incolor}5}]:} \PY{c+c1}{\PYZsh{} 生成数据集}
        \PY{n}{num\PYZus{}example} \PY{o}{=} \PY{l+m+mi}{2500}
        \PY{n}{x} \PY{o}{=}  \PY{n}{dbmoon}\PY{p}{(}\PY{n}{N}\PY{o}{=}\PY{n}{num\PYZus{}example} \PY{o}{/}\PY{o}{/} \PY{l+m+mi}{2}\PY{p}{)}
        \PY{n}{y} \PY{o}{=} \PY{n}{np}\PY{o}{.}\PY{n}{concatenate}\PY{p}{(}\PY{p}{[}\PY{n}{np}\PY{o}{.}\PY{n}{ones}\PY{p}{(}\PY{p}{[}\PY{n}{num\PYZus{}example} \PY{o}{/}\PY{o}{/} \PY{l+m+mi}{2}\PY{p}{]}\PY{p}{)}\PY{p}{,} \PY{o}{\PYZhy{}}\PY{n}{np}\PY{o}{.}\PY{n}{ones}\PY{p}{(}\PY{p}{[}\PY{n}{num\PYZus{}example} \PY{o}{/}\PY{o}{/} \PY{l+m+mi}{2}\PY{p}{]}\PY{p}{)}\PY{p}{]}\PY{p}{)}
        
        \PY{n}{plt}\PY{o}{.}\PY{n}{figure}\PY{p}{(}\PY{n}{figsize}\PY{o}{=}\PY{p}{(}\PY{l+m+mi}{8}\PY{p}{,} \PY{l+m+mi}{5}\PY{p}{)}\PY{p}{,} \PY{n}{dpi}\PY{o}{=}\PY{l+m+mi}{100}\PY{p}{)}
        \PY{n}{plt}\PY{o}{.}\PY{n}{scatter}\PY{p}{(}\PY{n}{x}\PY{p}{[}\PY{n}{num\PYZus{}example} \PY{o}{/}\PY{o}{/} \PY{l+m+mi}{2}\PY{p}{:}\PY{p}{,} \PY{l+m+mi}{0}\PY{p}{]}\PY{p}{,} \PY{n}{x}\PY{p}{[}\PY{n}{num\PYZus{}example} \PY{o}{/}\PY{o}{/} \PY{l+m+mi}{2}\PY{p}{:}\PY{p}{,} \PY{l+m+mi}{1}\PY{p}{]}\PY{p}{,} \PY{l+m+mi}{1}\PY{p}{,} \PY{n}{color}\PY{o}{=}\PY{l+s+s1}{\PYZsq{}}\PY{l+s+s1}{green}\PY{l+s+s1}{\PYZsq{}}\PY{p}{,} \PY{n}{label}\PY{o}{=}\PY{l+s+s1}{\PYZsq{}}\PY{l+s+s1}{Positive Class}\PY{l+s+s1}{\PYZsq{}}\PY{p}{)}
        \PY{n}{plt}\PY{o}{.}\PY{n}{legend}\PY{p}{(}\PY{n}{loc}\PY{o}{=}\PY{l+s+s1}{\PYZsq{}}\PY{l+s+s1}{best}\PY{l+s+s1}{\PYZsq{}}\PY{p}{)}
        \PY{n}{plt}\PY{o}{.}\PY{n}{scatter}\PY{p}{(}\PY{n}{x}\PY{p}{[}\PY{p}{:} \PY{n}{num\PYZus{}example} \PY{o}{/}\PY{o}{/} \PY{l+m+mi}{2}\PY{p}{,} \PY{l+m+mi}{0}\PY{p}{]}\PY{p}{,} \PY{n}{x}\PY{p}{[}\PY{p}{:} \PY{n}{num\PYZus{}example} \PY{o}{/}\PY{o}{/} \PY{l+m+mi}{2}\PY{p}{,} \PY{l+m+mi}{1}\PY{p}{]}\PY{p}{,} \PY{l+m+mi}{1}\PY{p}{,} \PY{n}{color}\PY{o}{=}\PY{l+s+s1}{\PYZsq{}}\PY{l+s+s1}{red}\PY{l+s+s1}{\PYZsq{}}\PY{p}{,} \PY{n}{label}\PY{o}{=}\PY{l+s+s1}{\PYZsq{}}\PY{l+s+s1}{Negative Class}\PY{l+s+s1}{\PYZsq{}}\PY{p}{)}
        \PY{n}{plt}\PY{o}{.}\PY{n}{legend}\PY{p}{(}\PY{n}{loc}\PY{o}{=}\PY{l+s+s1}{\PYZsq{}}\PY{l+s+s1}{best}\PY{l+s+s1}{\PYZsq{}}\PY{p}{)}
        \PY{n}{plt}\PY{o}{.}\PY{n}{show}\PY{p}{(}\PY{p}{)}
\end{Verbatim}


    \begin{center}
    \adjustimage{max size={0.9\linewidth}{0.9\paperheight}}{output_12_0.png}
    \end{center}
    { \hspace*{\fill} \\}
    
    我们将数据集打乱,并分成2000个训练集以及500个测试集。

    \begin{Verbatim}[commandchars=\\\{\}]
{\color{incolor}In [{\color{incolor}6}]:} \PY{k+kn}{from} \PY{n+nn}{sklearn}\PY{n+nn}{.}\PY{n+nn}{model\PYZus{}selection} \PY{k}{import} \PY{n}{train\PYZus{}test\PYZus{}split}
        
        \PY{c+c1}{\PYZsh{} 将数据集打乱并分为训练集与测试集}
        \PY{n}{train\PYZus{}x}\PY{p}{,} \PY{n}{test\PYZus{}x}\PY{p}{,} \PY{n}{train\PYZus{}y}\PY{p}{,} \PY{n}{test\PYZus{}y} \PY{o}{=} \PY{n}{train\PYZus{}test\PYZus{}split}\PY{p}{(}\PY{n}{x}\PY{p}{,} \PY{n}{y}\PY{p}{,} \PY{n}{test\PYZus{}size}\PY{o}{=}\PY{l+m+mi}{500}\PY{p}{)}
\end{Verbatim}


    到此数据集准备完成,接下来需要生成模型、代价函数以及训练部分的代码。分析数据集可以发现,每个数据点拥有2个属性,那么权重向量(不包括偏置值)的长度为2,所有首先我们可以确定模型的形式,如下:

    \begin{Verbatim}[commandchars=\\\{\}]
{\color{incolor}In [{\color{incolor}7}]:} \PY{n}{w} \PY{o}{=} \PY{n}{np}\PY{o}{.}\PY{n}{zeros}\PY{p}{(}\PY{p}{[}\PY{l+m+mi}{2}\PY{p}{]}\PY{p}{,} \PY{n}{dtype}\PY{o}{=}\PY{n}{np}\PY{o}{.}\PY{n}{float32}\PY{p}{)}
        \PY{n}{b} \PY{o}{=} \PY{l+m+mf}{0.}
        
        \PY{k}{def} \PY{n+nf}{rosenblatt}\PY{p}{(}\PY{n}{x}\PY{p}{)}\PY{p}{:}
            \PY{n}{z} \PY{o}{=} \PY{n}{np}\PY{o}{.}\PY{n}{sum}\PY{p}{(}\PY{n}{w} \PY{o}{*} \PY{n}{x}\PY{p}{)} \PY{o}{+} \PY{n}{b}
            \PY{k}{if} \PY{n}{z} \PY{o}{\PYZgt{}}\PY{o}{=} \PY{l+m+mi}{0}\PY{p}{:}
                \PY{k}{return} \PY{l+m+mi}{1}
            \PY{k}{else}\PY{p}{:}
                \PY{k}{return} \PY{o}{\PYZhy{}}\PY{l+m+mi}{1}
\end{Verbatim}


    模型的代价函数为均方误差代价函数,\(MSE=\frac{1}{N}\sum_{i=1}^{N}(label_i-predicted_i)^2\),实现如下:

    \begin{Verbatim}[commandchars=\\\{\}]
{\color{incolor}In [{\color{incolor}8}]:} \PY{k}{def} \PY{n+nf}{mse}\PY{p}{(}\PY{n}{label}\PY{p}{,} \PY{n}{pred}\PY{p}{)}\PY{p}{:}
            \PY{k}{return} \PY{n}{np}\PY{o}{.}\PY{n}{average}\PY{p}{(}\PY{p}{(}\PY{n}{np}\PY{o}{.}\PY{n}{array}\PY{p}{(}\PY{n}{label}\PY{p}{)} \PY{o}{\PYZhy{}} \PY{n}{np}\PY{o}{.}\PY{n}{array}\PY{p}{(}\PY{n}{pred}\PY{p}{)}\PY{p}{)} \PY{o}{*}\PY{o}{*} \PY{l+m+mi}{2}\PY{p}{)}
\end{Verbatim}


    当然,也可使用sklearn中的评价指标,即\texttt{sklearn.metrics}模块下的\texttt{mean\_squared\_error}实现代价函数。

    接下来,我们就可以根据感知器收敛算法训练模型。为了能够清晰的了解训练过程,我们在每次训练之后使用测试集评估模型,并记录下来。最后将最终的决策边界绘制在图中。

    \begin{Verbatim}[commandchars=\\\{\}]
{\color{incolor}In [{\color{incolor}9}]:} \PY{c+c1}{\PYZsh{} 训练模型 10000 次}
        \PY{n}{train\PYZus{}steps} \PY{o}{=} \PY{l+m+mi}{10000}
        \PY{c+c1}{\PYZsh{} 摘要频率}
        \PY{n}{summary\PYZus{}step} \PY{o}{=} \PY{l+m+mi}{50}
        \PY{c+c1}{\PYZsh{} 摘要,记录训练中的代价变化}
        \PY{n}{summary} \PY{o}{=} \PY{n}{np}\PY{o}{.}\PY{n}{empty}\PY{p}{(}\PY{p}{[}\PY{n}{train\PYZus{}steps} \PY{o}{/}\PY{o}{/} \PY{n}{summary\PYZus{}step}\PY{p}{,} \PY{l+m+mi}{2}\PY{p}{]}\PY{p}{)}
        \PY{c+c1}{\PYZsh{} 学习率}
        \PY{n}{lr} \PY{o}{=} \PY{l+m+mf}{0.1}
        
        \PY{k}{for} \PY{n}{i} \PY{o+ow}{in} \PY{n+nb}{range}\PY{p}{(}\PY{l+m+mi}{0}\PY{p}{,} \PY{n}{train\PYZus{}steps}\PY{p}{)}\PY{p}{:}
            \PY{c+c1}{\PYZsh{} 评估模型}
            \PY{k}{if} \PY{n}{i} \PY{o}{\PYZpc{}} \PY{n}{summary\PYZus{}step} \PY{o}{==} \PY{l+m+mi}{0}\PY{p}{:}
                \PY{n}{test\PYZus{}out} \PY{o}{=} \PY{p}{[}\PY{p}{]}
                \PY{k}{for} \PY{n}{j} \PY{o+ow}{in} \PY{n+nb}{range}\PY{p}{(}\PY{n}{test\PYZus{}y}\PY{o}{.}\PY{n}{shape}\PY{p}{[}\PY{l+m+mi}{0}\PY{p}{]}\PY{p}{)}\PY{p}{:}
                    \PY{n}{test\PYZus{}out}\PY{o}{.}\PY{n}{append}\PY{p}{(}\PY{n}{rosenblatt}\PY{p}{(}\PY{n}{test\PYZus{}x}\PY{p}{[}\PY{n}{j}\PY{p}{]}\PY{p}{)}\PY{p}{)}
                \PY{n}{loss} \PY{o}{=} \PY{n}{mse}\PY{p}{(}\PY{n}{test\PYZus{}y}\PY{p}{,} \PY{n}{test\PYZus{}out}\PY{p}{)}
                \PY{n}{idx} \PY{o}{=} \PY{n+nb}{int}\PY{p}{(}\PY{n}{i} \PY{o}{/} \PY{n}{summary\PYZus{}step}\PY{p}{)}
                \PY{n}{summary}\PY{p}{[}\PY{n}{idx}\PY{p}{]} \PY{o}{=} \PY{n}{np}\PY{o}{.}\PY{n}{array}\PY{p}{(}\PY{p}{[}\PY{n}{i}\PY{p}{,} \PY{n}{loss}\PY{p}{]}\PY{p}{)}
            
            \PY{c+c1}{\PYZsh{} 取一个训练集中的样本}
            \PY{n}{one\PYZus{}x}\PY{p}{,} \PY{n}{one\PYZus{}y} \PY{o}{=} \PY{n}{train\PYZus{}x}\PY{p}{[}\PY{n}{i} \PY{o}{\PYZpc{}} \PY{n}{train\PYZus{}y}\PY{o}{.}\PY{n}{shape}\PY{p}{[}\PY{l+m+mi}{0}\PY{p}{]}\PY{p}{]}\PY{p}{,} \PY{n}{train\PYZus{}y}\PY{p}{[}\PY{n}{i} \PY{o}{\PYZpc{}} \PY{n}{train\PYZus{}y}\PY{o}{.}\PY{n}{shape}\PY{p}{[}\PY{l+m+mi}{0}\PY{p}{]}\PY{p}{]}
            \PY{c+c1}{\PYZsh{} 得到模型输出结果}
            \PY{n}{out} \PY{o}{=} \PY{n}{rosenblatt}\PY{p}{(}\PY{n}{one\PYZus{}x}\PY{p}{)}
            \PY{c+c1}{\PYZsh{} 更新权值}
            \PY{n}{w} \PY{o}{=} \PY{n}{w} \PY{o}{+} \PY{n}{lr} \PY{o}{*} \PY{p}{(}\PY{n}{one\PYZus{}y} \PY{o}{\PYZhy{}} \PY{n}{out}\PY{p}{)} \PY{o}{*} \PY{n}{one\PYZus{}x}
            \PY{n}{b} \PY{o}{=} \PY{n}{b} \PY{o}{+} \PY{n}{lr} \PY{o}{*} \PY{p}{(}\PY{n}{one\PYZus{}y} \PY{o}{\PYZhy{}} \PY{n}{out}\PY{p}{)}
\end{Verbatim}


    训练完成之后,我们将训练结果进行可视化。

    \begin{Verbatim}[commandchars=\\\{\}]
{\color{incolor}In [{\color{incolor}10}]:} \PY{n}{plt}\PY{o}{.}\PY{n}{figure}\PY{p}{(}\PY{n}{figsize}\PY{o}{=}\PY{p}{(}\PY{l+m+mi}{9}\PY{p}{,} \PY{l+m+mi}{3}\PY{p}{)}\PY{p}{,} \PY{n}{dpi}\PY{o}{=}\PY{l+m+mi}{100}\PY{p}{)}
         
         \PY{n}{plt}\PY{o}{.}\PY{n}{subplot}\PY{p}{(}\PY{l+m+mi}{1}\PY{p}{,} \PY{l+m+mi}{2}\PY{p}{,} \PY{l+m+mi}{1}\PY{p}{)}
         \PY{n}{plt}\PY{o}{.}\PY{n}{plot}\PY{p}{(}\PY{n}{summary}\PY{p}{[}\PY{p}{:}\PY{p}{,} \PY{l+m+mi}{0}\PY{p}{]}\PY{p}{,} \PY{n}{summary}\PY{p}{[}\PY{p}{:}\PY{p}{,} \PY{l+m+mi}{1}\PY{p}{]}\PY{p}{,} \PY{n}{label}\PY{o}{=}\PY{l+s+s1}{\PYZsq{}}\PY{l+s+s1}{MSE}\PY{l+s+s1}{\PYZsq{}}\PY{p}{)}
         \PY{n}{plt}\PY{o}{.}\PY{n}{legend}\PY{p}{(}\PY{n}{loc}\PY{o}{=}\PY{l+s+s1}{\PYZsq{}}\PY{l+s+s1}{best}\PY{l+s+s1}{\PYZsq{}}\PY{p}{)}
         \PY{n}{plt}\PY{o}{.}\PY{n}{xlabel}\PY{p}{(}\PY{l+s+s1}{\PYZsq{}}\PY{l+s+s1}{迭代次数}\PY{l+s+s1}{\PYZsq{}}\PY{p}{,} \PY{n}{fontproperties}\PY{o}{=}\PY{l+s+s1}{\PYZsq{}}\PY{l+s+s1}{SimHei}\PY{l+s+s1}{\PYZsq{}}\PY{p}{)}
         \PY{n}{plt}\PY{o}{.}\PY{n}{ylabel}\PY{p}{(}\PY{l+s+s1}{\PYZsq{}}\PY{l+s+s1}{均方误差}\PY{l+s+s1}{\PYZsq{}}\PY{p}{,} \PY{n}{fontproperties}\PY{o}{=}\PY{l+s+s1}{\PYZsq{}}\PY{l+s+s1}{SimHei}\PY{l+s+s1}{\PYZsq{}}\PY{p}{)}
         
         \PY{n}{plt}\PY{o}{.}\PY{n}{subplot}\PY{p}{(}\PY{l+m+mi}{1}\PY{p}{,} \PY{l+m+mi}{2}\PY{p}{,} \PY{l+m+mi}{2}\PY{p}{)}
         \PY{n}{plt}\PY{o}{.}\PY{n}{scatter}\PY{p}{(}\PY{n}{x}\PY{p}{[}\PY{n}{num\PYZus{}example} \PY{o}{/}\PY{o}{/} \PY{l+m+mi}{2}\PY{p}{:}\PY{p}{,} \PY{l+m+mi}{0}\PY{p}{]}\PY{p}{,} \PY{n}{x}\PY{p}{[}\PY{n}{num\PYZus{}example} \PY{o}{/}\PY{o}{/} \PY{l+m+mi}{2}\PY{p}{:}\PY{p}{,} \PY{l+m+mi}{1}\PY{p}{]}\PY{p}{,} \PY{l+m+mi}{1}\PY{p}{,} \PY{n}{color}\PY{o}{=}\PY{l+s+s1}{\PYZsq{}}\PY{l+s+s1}{green}\PY{l+s+s1}{\PYZsq{}}\PY{p}{,} \PY{n}{label}\PY{o}{=}\PY{l+s+s1}{\PYZsq{}}\PY{l+s+s1}{Pos}\PY{l+s+s1}{\PYZsq{}}\PY{p}{)}
         \PY{n}{plt}\PY{o}{.}\PY{n}{legend}\PY{p}{(}\PY{n}{loc}\PY{o}{=}\PY{l+s+s1}{\PYZsq{}}\PY{l+s+s1}{best}\PY{l+s+s1}{\PYZsq{}}\PY{p}{)}
         \PY{n}{plt}\PY{o}{.}\PY{n}{scatter}\PY{p}{(}\PY{n}{x}\PY{p}{[}\PY{p}{:} \PY{n}{num\PYZus{}example} \PY{o}{/}\PY{o}{/} \PY{l+m+mi}{2}\PY{p}{,} \PY{l+m+mi}{0}\PY{p}{]}\PY{p}{,} \PY{n}{x}\PY{p}{[}\PY{p}{:} \PY{n}{num\PYZus{}example} \PY{o}{/}\PY{o}{/} \PY{l+m+mi}{2}\PY{p}{,} \PY{l+m+mi}{1}\PY{p}{]}\PY{p}{,} \PY{l+m+mi}{1}\PY{p}{,} \PY{n}{color}\PY{o}{=}\PY{l+s+s1}{\PYZsq{}}\PY{l+s+s1}{red}\PY{l+s+s1}{\PYZsq{}}\PY{p}{,} \PY{n}{label}\PY{o}{=}\PY{l+s+s1}{\PYZsq{}}\PY{l+s+s1}{Neg}\PY{l+s+s1}{\PYZsq{}}\PY{p}{)}
         \PY{n}{plt}\PY{o}{.}\PY{n}{legend}\PY{p}{(}\PY{n}{loc}\PY{o}{=}\PY{l+s+s1}{\PYZsq{}}\PY{l+s+s1}{best}\PY{l+s+s1}{\PYZsq{}}\PY{p}{)}
         \PY{n}{left\PYZus{}coord\PYZus{}y} \PY{o}{=} \PY{p}{(}\PY{o}{\PYZhy{}}\PY{n}{b} \PY{o}{\PYZhy{}} \PY{p}{(}\PY{o}{\PYZhy{}}\PY{l+m+mi}{12}\PY{p}{)} \PY{o}{*} \PY{n}{w}\PY{p}{[}\PY{l+m+mi}{0}\PY{p}{]}\PY{p}{)} \PY{o}{/} \PY{n}{w}\PY{p}{[}\PY{l+m+mi}{1}\PY{p}{]}
         \PY{n}{right\PYZus{}coord\PYZus{}y} \PY{o}{=} \PY{p}{(}\PY{o}{\PYZhy{}}\PY{n}{b} \PY{o}{\PYZhy{}} \PY{p}{(}\PY{l+m+mi}{22}\PY{p}{)} \PY{o}{*} \PY{n}{w}\PY{p}{[}\PY{l+m+mi}{0}\PY{p}{]}\PY{p}{)} \PY{o}{/} \PY{n}{w}\PY{p}{[}\PY{l+m+mi}{1}\PY{p}{]}
         \PY{n}{plt}\PY{o}{.}\PY{n}{plot}\PY{p}{(}\PY{p}{[}\PY{o}{\PYZhy{}}\PY{l+m+mi}{12}\PY{p}{,} \PY{l+m+mi}{22}\PY{p}{]}\PY{p}{,} \PY{p}{[}\PY{n}{left\PYZus{}coord\PYZus{}y}\PY{p}{,} \PY{n}{right\PYZus{}coord\PYZus{}y}\PY{p}{]}\PY{p}{,} \PY{n}{linewidth}\PY{o}{=}\PY{l+m+mi}{2}\PY{p}{)}
         
         \PY{n}{plt}\PY{o}{.}\PY{n}{annotate}\PY{p}{(}
             \PY{l+s+s1}{\PYZsq{}}\PY{l+s+s1}{决策边界}\PY{l+s+s1}{\PYZsq{}}\PY{p}{,} \PY{n}{xy}\PY{o}{=}\PY{p}{(}\PY{o}{\PYZhy{}}\PY{l+m+mi}{2}\PY{p}{,} \PY{o}{\PYZhy{}}\PY{l+m+mf}{0.2}\PY{p}{)}\PY{p}{,} \PY{n}{xytext}\PY{o}{=}\PY{p}{(}\PY{o}{\PYZhy{}}\PY{l+m+mf}{10.}\PY{p}{,} \PY{o}{\PYZhy{}}\PY{l+m+mf}{6.5}\PY{p}{)}\PY{p}{,} 
              \PY{n}{arrowprops}\PY{o}{=}\PY{n+nb}{dict}\PY{p}{(}\PY{n}{arrowstyle}\PY{o}{=}\PY{l+s+s2}{\PYZdq{}}\PY{l+s+s2}{\PYZhy{}\PYZgt{}}\PY{l+s+s2}{\PYZdq{}}\PY{p}{,} \PY{n}{connectionstyle}\PY{o}{=}\PY{l+s+s2}{\PYZdq{}}\PY{l+s+s2}{arc3,rad=\PYZhy{}.2}\PY{l+s+s2}{\PYZdq{}}\PY{p}{)}\PY{p}{,} 
              \PY{n}{fontproperties}\PY{o}{=}\PY{l+s+s1}{\PYZsq{}}\PY{l+s+s1}{SimHei}\PY{l+s+s1}{\PYZsq{}}\PY{p}{,} \PY{n}{fontsize}\PY{o}{=}\PY{l+m+mi}{12}\PY{p}{)}
         
         \PY{n}{plt}\PY{o}{.}\PY{n}{show}\PY{p}{(}\PY{p}{)}
\end{Verbatim}


    \begin{center}
    \adjustimage{max size={0.9\linewidth}{0.9\paperheight}}{output_23_0.png}
    \end{center}
    { \hspace*{\fill} \\}
    
    当两个半月之间的垂直距离变为负数时,感知器无法收敛,其在训练的过程中代价不断波动。

    \section{2.4
对于线性不可分数据}\label{ux5bf9ux4e8eux7ebfux6027ux4e0dux53efux5206ux6570ux636e}

在线性不可分数据集中,Rosenblatt感知器也并非完全无法使用,这时候可以对输入模型的特征做一系列非线性的变换进而得到新的特征,这些特征在Rosenblatt感知器可能是线性可分的。

    例如,对于如下数据集,直接使用Rosenblatt感知器无法分类。

    \begin{Verbatim}[commandchars=\\\{\}]
{\color{incolor}In [{\color{incolor}12}]:} \PY{k+kn}{from} \PY{n+nn}{sklearn}\PY{n+nn}{.}\PY{n+nn}{datasets} \PY{k}{import} \PY{n}{make\PYZus{}circles}
         
         \PY{n}{plt}\PY{o}{.}\PY{n}{figure}\PY{p}{(}\PY{n}{figsize}\PY{o}{=}\PY{p}{(}\PY{l+m+mi}{3}\PY{p}{,} \PY{l+m+mi}{3}\PY{p}{)}\PY{p}{,} \PY{n}{dpi}\PY{o}{=}\PY{l+m+mi}{100}\PY{p}{)}
         \PY{n}{circles\PYZus{}data} \PY{o}{=} \PY{n}{make\PYZus{}circles}\PY{p}{(}\PY{n}{shuffle}\PY{o}{=}\PY{k+kc}{False}\PY{p}{,} \PY{n}{noise}\PY{o}{=}\PY{l+m+mf}{0.05}\PY{p}{,} \PY{n}{factor}\PY{o}{=}\PY{l+m+mf}{0.3}\PY{p}{)}
         \PY{n}{circles\PYZus{}data}\PY{p}{[}\PY{l+m+mi}{1}\PY{p}{]}\PY{p}{[}\PY{p}{:}\PY{l+m+mi}{50}\PY{p}{]} \PY{o}{=} \PY{o}{\PYZhy{}}\PY{l+m+mi}{1}
         \PY{n}{plt}\PY{o}{.}\PY{n}{scatter}\PY{p}{(}\PY{n}{circles\PYZus{}data}\PY{p}{[}\PY{l+m+mi}{0}\PY{p}{]}\PY{p}{[}\PY{p}{:}\PY{l+m+mi}{50}\PY{p}{,}\PY{l+m+mi}{0}\PY{p}{]}\PY{p}{,} \PY{n}{circles\PYZus{}data}\PY{p}{[}\PY{l+m+mi}{0}\PY{p}{]}\PY{p}{[}\PY{p}{:}\PY{l+m+mi}{50}\PY{p}{,}\PY{l+m+mi}{1}\PY{p}{]}\PY{p}{,} \PY{l+m+mi}{10}\PY{p}{,} \PY{n}{color}\PY{o}{=}\PY{l+s+s1}{\PYZsq{}}\PY{l+s+s1}{blue}\PY{l+s+s1}{\PYZsq{}}\PY{p}{)}
         \PY{n}{plt}\PY{o}{.}\PY{n}{scatter}\PY{p}{(}\PY{n}{circles\PYZus{}data}\PY{p}{[}\PY{l+m+mi}{0}\PY{p}{]}\PY{p}{[}\PY{l+m+mi}{50}\PY{p}{:}\PY{p}{,}\PY{l+m+mi}{0}\PY{p}{]}\PY{p}{,} \PY{n}{circles\PYZus{}data}\PY{p}{[}\PY{l+m+mi}{0}\PY{p}{]}\PY{p}{[}\PY{l+m+mi}{50}\PY{p}{:}\PY{p}{,}\PY{l+m+mi}{1}\PY{p}{]}\PY{p}{,} \PY{l+m+mi}{10}\PY{p}{,} \PY{n}{color}\PY{o}{=}\PY{l+s+s1}{\PYZsq{}}\PY{l+s+s1}{yellow}\PY{l+s+s1}{\PYZsq{}}\PY{p}{)}
\end{Verbatim}


\begin{Verbatim}[commandchars=\\\{\}]
{\color{outcolor}Out[{\color{outcolor}12}]:} <matplotlib.collections.PathCollection at 0x1112278d0>
\end{Verbatim}
            
    \begin{center}
    \adjustimage{max size={0.9\linewidth}{0.9\paperheight}}{output_27_1.png}
    \end{center}
    { \hspace*{\fill} \\}
    
    这时候可以使用数据集的属性的非线性变换对数据进行变换,使得可以使用感知器进行分类。这里我们可以使用每个属性的平方来进行,如下:

    \begin{Verbatim}[commandchars=\\\{\}]
{\color{incolor}In [{\color{incolor}13}]:} \PY{c+c1}{\PYZsh{} 将数据集打乱并分为训练集与测试集}
         \PY{n}{train\PYZus{}x}\PY{p}{,} \PY{n}{test\PYZus{}x}\PY{p}{,} \PY{n}{train\PYZus{}y}\PY{p}{,} \PY{n}{test\PYZus{}y} \PY{o}{=} \PY{n}{train\PYZus{}test\PYZus{}split}\PY{p}{(}\PY{n}{circles\PYZus{}data}\PY{p}{[}\PY{l+m+mi}{0}\PY{p}{]}\PY{p}{,} \PY{n}{circles\PYZus{}data}\PY{p}{[}\PY{l+m+mi}{1}\PY{p}{]}\PY{p}{,} \PY{n}{test\PYZus{}size}\PY{o}{=}\PY{l+m+mi}{20}\PY{p}{)}
         
         \PY{c+c1}{\PYZsh{} 训练模型 1000 次}
         \PY{n}{train\PYZus{}steps} \PY{o}{=} \PY{l+m+mi}{100}
         \PY{c+c1}{\PYZsh{} 摘要频率}
         \PY{n}{summary\PYZus{}step} \PY{o}{=} \PY{l+m+mi}{1}
         \PY{c+c1}{\PYZsh{} 摘要,记录训练中的代价变化}
         \PY{n}{summary} \PY{o}{=} \PY{n}{np}\PY{o}{.}\PY{n}{empty}\PY{p}{(}\PY{p}{[}\PY{n}{train\PYZus{}steps} \PY{o}{/}\PY{o}{/} \PY{n}{summary\PYZus{}step}\PY{p}{,} \PY{l+m+mi}{2}\PY{p}{]}\PY{p}{)}
         \PY{c+c1}{\PYZsh{} 学习率}
         \PY{n}{lr} \PY{o}{=} \PY{l+m+mf}{0.01}
         
         \PY{n}{w} \PY{o}{=} \PY{n}{np}\PY{o}{.}\PY{n}{zeros}\PY{p}{(}\PY{p}{[}\PY{l+m+mi}{2}\PY{p}{]}\PY{p}{,} \PY{n}{dtype}\PY{o}{=}\PY{n}{np}\PY{o}{.}\PY{n}{float32}\PY{p}{)}
         \PY{n}{b} \PY{o}{=} \PY{l+m+mf}{0.}
         
         \PY{k}{def} \PY{n+nf}{rosenblatt}\PY{p}{(}\PY{n}{x}\PY{p}{)}\PY{p}{:}
             \PY{n}{z} \PY{o}{=} \PY{n}{np}\PY{o}{.}\PY{n}{sum}\PY{p}{(}\PY{n}{w} \PY{o}{*} \PY{n}{x}\PY{p}{)} \PY{o}{+} \PY{n}{b}
             \PY{k}{if} \PY{n}{z} \PY{o}{\PYZgt{}}\PY{o}{=} \PY{l+m+mi}{0}\PY{p}{:}
                 \PY{k}{return} \PY{l+m+mi}{1}
             \PY{k}{else}\PY{p}{:}
                 \PY{k}{return} \PY{o}{\PYZhy{}}\PY{l+m+mi}{1}
         
         \PY{k}{for} \PY{n}{i} \PY{o+ow}{in} \PY{n+nb}{range}\PY{p}{(}\PY{l+m+mi}{0}\PY{p}{,} \PY{n}{train\PYZus{}steps}\PY{p}{)}\PY{p}{:}
             \PY{c+c1}{\PYZsh{} 评估模型}
             \PY{k}{if} \PY{n}{i} \PY{o}{\PYZpc{}} \PY{n}{summary\PYZus{}step} \PY{o}{==} \PY{l+m+mi}{0}\PY{p}{:}
                 \PY{n}{test\PYZus{}out} \PY{o}{=} \PY{p}{[}\PY{p}{]}
                 \PY{k}{for} \PY{n}{j} \PY{o+ow}{in} \PY{n+nb}{range}\PY{p}{(}\PY{n}{test\PYZus{}y}\PY{o}{.}\PY{n}{shape}\PY{p}{[}\PY{l+m+mi}{0}\PY{p}{]}\PY{p}{)}\PY{p}{:}
                     \PY{n}{test\PYZus{}out}\PY{o}{.}\PY{n}{append}\PY{p}{(}\PY{n}{rosenblatt}\PY{p}{(}\PY{n}{test\PYZus{}x}\PY{p}{[}\PY{n}{j}\PY{p}{]} \PY{o}{*}\PY{o}{*} \PY{l+m+mi}{2}\PY{p}{)}\PY{p}{)}
                 \PY{n}{loss} \PY{o}{=} \PY{n}{mse}\PY{p}{(}\PY{n}{test\PYZus{}y}\PY{p}{,} \PY{n}{test\PYZus{}out}\PY{p}{)}
                 \PY{n}{idx} \PY{o}{=} \PY{n+nb}{int}\PY{p}{(}\PY{n}{i} \PY{o}{/} \PY{n}{summary\PYZus{}step}\PY{p}{)}
                 \PY{n}{summary}\PY{p}{[}\PY{n}{idx}\PY{p}{]} \PY{o}{=} \PY{n}{np}\PY{o}{.}\PY{n}{array}\PY{p}{(}\PY{p}{[}\PY{n}{i}\PY{p}{,} \PY{n}{loss}\PY{p}{]}\PY{p}{)}
             
             \PY{c+c1}{\PYZsh{} 取一个训练集中的样本}
             \PY{n}{one\PYZus{}x}\PY{p}{,} \PY{n}{one\PYZus{}y} \PY{o}{=} \PY{n}{train\PYZus{}x}\PY{p}{[}\PY{n}{i} \PY{o}{\PYZpc{}} \PY{n}{train\PYZus{}y}\PY{o}{.}\PY{n}{shape}\PY{p}{[}\PY{l+m+mi}{0}\PY{p}{]}\PY{p}{]}\PY{p}{,} \PY{n}{train\PYZus{}y}\PY{p}{[}\PY{n}{i} \PY{o}{\PYZpc{}} \PY{n}{train\PYZus{}y}\PY{o}{.}\PY{n}{shape}\PY{p}{[}\PY{l+m+mi}{0}\PY{p}{]}\PY{p}{]}
             \PY{c+c1}{\PYZsh{} 得到模型输出结果}
             \PY{n}{out} \PY{o}{=} \PY{n}{rosenblatt}\PY{p}{(}\PY{n}{one\PYZus{}x} \PY{o}{*}\PY{o}{*} \PY{l+m+mi}{2}\PY{p}{)}
             \PY{c+c1}{\PYZsh{} 更新权值}
             \PY{n}{w} \PY{o}{=} \PY{n}{w} \PY{o}{+} \PY{n}{lr} \PY{o}{*} \PY{p}{(}\PY{n}{one\PYZus{}y} \PY{o}{\PYZhy{}} \PY{n}{out}\PY{p}{)} \PY{o}{*} \PY{p}{(}\PY{n}{one\PYZus{}x} \PY{o}{*}\PY{o}{*} \PY{l+m+mi}{2}\PY{p}{)}
             \PY{n}{b} \PY{o}{=} \PY{n}{b} \PY{o}{+} \PY{n}{lr} \PY{o}{*} \PY{p}{(}\PY{n}{one\PYZus{}y} \PY{o}{\PYZhy{}} \PY{n}{out}\PY{p}{)} 
\end{Verbatim}


    \begin{Verbatim}[commandchars=\\\{\}]
{\color{incolor}In [{\color{incolor}15}]:} \PY{n}{plt}\PY{o}{.}\PY{n}{figure}\PY{p}{(}\PY{n}{figsize}\PY{o}{=}\PY{p}{(}\PY{l+m+mi}{8}\PY{p}{,} \PY{l+m+mi}{3}\PY{p}{)}\PY{p}{,} \PY{n}{dpi}\PY{o}{=}\PY{l+m+mi}{120}\PY{p}{)}
         
         \PY{n}{plt}\PY{o}{.}\PY{n}{plot}\PY{p}{(}\PY{n}{summary}\PY{p}{[}\PY{p}{:}\PY{p}{,} \PY{l+m+mi}{0}\PY{p}{]}\PY{p}{,} \PY{n}{summary}\PY{p}{[}\PY{p}{:}\PY{p}{,} \PY{l+m+mi}{1}\PY{p}{]}\PY{p}{,} \PY{n}{label}\PY{o}{=}\PY{l+s+s1}{\PYZsq{}}\PY{l+s+s1}{MSE}\PY{l+s+s1}{\PYZsq{}}\PY{p}{)}
         \PY{n}{plt}\PY{o}{.}\PY{n}{legend}\PY{p}{(}\PY{n}{loc}\PY{o}{=}\PY{l+s+s1}{\PYZsq{}}\PY{l+s+s1}{best}\PY{l+s+s1}{\PYZsq{}}\PY{p}{)}
         \PY{n}{plt}\PY{o}{.}\PY{n}{xlabel}\PY{p}{(}\PY{l+s+s1}{\PYZsq{}}\PY{l+s+s1}{迭代次数}\PY{l+s+s1}{\PYZsq{}}\PY{p}{,} \PY{n}{fontproperties}\PY{o}{=}\PY{l+s+s1}{\PYZsq{}}\PY{l+s+s1}{SimHei}\PY{l+s+s1}{\PYZsq{}}\PY{p}{)}
         \PY{n}{plt}\PY{o}{.}\PY{n}{ylabel}\PY{p}{(}\PY{l+s+s1}{\PYZsq{}}\PY{l+s+s1}{均方误差}\PY{l+s+s1}{\PYZsq{}}\PY{p}{,} \PY{n}{fontproperties}\PY{o}{=}\PY{l+s+s1}{\PYZsq{}}\PY{l+s+s1}{SimHei}\PY{l+s+s1}{\PYZsq{}}\PY{p}{)}
         
         \PY{n}{plt}\PY{o}{.}\PY{n}{show}\PY{p}{(}\PY{p}{)}
\end{Verbatim}


    \begin{center}
    \adjustimage{max size={0.9\linewidth}{0.9\paperheight}}{output_30_0.png}
    \end{center}
    { \hspace*{\fill} \\}
    
    使用属性组合只有在少数情况下是适用,在属性较多时,属性的不同组合将来指数级的新属性,使得模型变得很复杂。

    \section{练习}\label{ux7ec3ux4e60}

\begin{enumerate}
\def\labelenumi{\arabic{enumi}.}
\item
  代码练习

  \begin{itemize}
  \item
    尝试生成垂直距离小于零的双月数据集,然后使用此数据集训练Rosenblatt感知器模型,记录其代价变化,并可视化。
  \item
    思考:根据上述训练结果以及决策边界的可视化结果,说明训练Rosenblatt感知器去解决线性不可分问题是否意味着模型完全失效?如果不是,思考在训练时如何确定训练终止条件。
  \item
    训练感知器模型时,使用到了学习率这个超参数,实际中,学习率是较难确定的超参数之一,为此提出了学习率衰减,请自行学习学习率衰减的原理与实现方法,并实现线性衰减学习率模型应用与Rosenblatt感知器模型的训练中。
  \item
    在训练模型时,模型训练何时终止并无说明,但由于Rosenblatt感知器模型在线性可分的数据集中一定可以收敛,所以代价会逐步趋于零或等于零,此时即可终止训练,尝试使用代码实现动态终止训练。
  \end{itemize}
\item
  判断题

  \begin{itemize}
  \tightlist
  \item
    {[} {]} Rosenblatt感知器只能训练其进行二类别的判别。
  \item
    {[}x{]} 对于线性可分的数据集,Rosenblatt感知器一定能够收敛。
  \item
    {[}x{]}
    Rosenblatt(1958)提出感知器作为神经网络中监督学习的第一个模型。
  \item
    {[} {]} 一般的在二分类任务中,Rosenblatt感知器的激活函数输出 0
    表示负类,输出 1 表示正类。
  \item
    {[}x{]} 均方误差代价函数可用于衡量模型在分类任务中的性能。
  \end{itemize}
\end{enumerate}


    % Add a bibliography block to the postdoc
    
    
    
    \end{document}
